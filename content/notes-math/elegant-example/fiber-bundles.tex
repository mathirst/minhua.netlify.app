% Options for packages loaded elsewhere
\PassOptionsToPackage{unicode}{hyperref}
\PassOptionsToPackage{hyphens}{url}
\PassOptionsToPackage{dvipsnames,svgnames,x11names}{xcolor}
%
\documentclass[
  lang=cn,
  titlestyle=hang,
  chinesefont=ctexfont]{elegantbook}
\usepackage{amsmath,amssymb}
\usepackage{lmodern}
\usepackage{iftex}
\ifPDFTeX
  \usepackage[T1]{fontenc}
  \usepackage[utf8]{inputenc}
  \usepackage{textcomp} % provide euro and other symbols
\else % if luatex or xetex
  \ifXeTeX
    \usepackage{mathspec}
  \else
    \usepackage{unicode-math}
  \fi
  \defaultfontfeatures{Scale=MatchLowercase}
  \defaultfontfeatures[\rmfamily]{Ligatures=TeX,Scale=1}
\fi
% Use upquote if available, for straight quotes in verbatim environments
\IfFileExists{upquote.sty}{\usepackage{upquote}}{}
\IfFileExists{microtype.sty}{% use microtype if available
  \usepackage[]{microtype}
  \UseMicrotypeSet[protrusion]{basicmath} % disable protrusion for tt fonts
}{}
\makeatletter
\@ifundefined{KOMAClassName}{% if non-KOMA class
  \IfFileExists{parskip.sty}{%
    \usepackage{parskip}
  }{% else
    \setlength{\parindent}{0pt}
    \setlength{\parskip}{6pt plus 2pt minus 1pt}}
}{% if KOMA class
  \KOMAoptions{parskip=half}}
\makeatother
\usepackage{xcolor}
\usepackage{color}
\usepackage{fancyvrb}
\newcommand{\VerbBar}{|}
\newcommand{\VERB}{\Verb[commandchars=\\\{\}]}
\DefineVerbatimEnvironment{Highlighting}{Verbatim}{commandchars=\\\{\}}
% Add ',fontsize=\small' for more characters per line
\usepackage{framed}
\definecolor{shadecolor}{RGB}{248,248,248}
\newenvironment{Shaded}{\begin{snugshade}}{\end{snugshade}}
\newcommand{\AlertTok}[1]{\textcolor[rgb]{0.94,0.16,0.16}{#1}}
\newcommand{\AnnotationTok}[1]{\textcolor[rgb]{0.56,0.35,0.01}{\textbf{\textit{#1}}}}
\newcommand{\AttributeTok}[1]{\textcolor[rgb]{0.77,0.63,0.00}{#1}}
\newcommand{\BaseNTok}[1]{\textcolor[rgb]{0.00,0.00,0.81}{#1}}
\newcommand{\BuiltInTok}[1]{#1}
\newcommand{\CharTok}[1]{\textcolor[rgb]{0.31,0.60,0.02}{#1}}
\newcommand{\CommentTok}[1]{\textcolor[rgb]{0.56,0.35,0.01}{\textit{#1}}}
\newcommand{\CommentVarTok}[1]{\textcolor[rgb]{0.56,0.35,0.01}{\textbf{\textit{#1}}}}
\newcommand{\ConstantTok}[1]{\textcolor[rgb]{0.00,0.00,0.00}{#1}}
\newcommand{\ControlFlowTok}[1]{\textcolor[rgb]{0.13,0.29,0.53}{\textbf{#1}}}
\newcommand{\DataTypeTok}[1]{\textcolor[rgb]{0.13,0.29,0.53}{#1}}
\newcommand{\DecValTok}[1]{\textcolor[rgb]{0.00,0.00,0.81}{#1}}
\newcommand{\DocumentationTok}[1]{\textcolor[rgb]{0.56,0.35,0.01}{\textbf{\textit{#1}}}}
\newcommand{\ErrorTok}[1]{\textcolor[rgb]{0.64,0.00,0.00}{\textbf{#1}}}
\newcommand{\ExtensionTok}[1]{#1}
\newcommand{\FloatTok}[1]{\textcolor[rgb]{0.00,0.00,0.81}{#1}}
\newcommand{\FunctionTok}[1]{\textcolor[rgb]{0.00,0.00,0.00}{#1}}
\newcommand{\ImportTok}[1]{#1}
\newcommand{\InformationTok}[1]{\textcolor[rgb]{0.56,0.35,0.01}{\textbf{\textit{#1}}}}
\newcommand{\KeywordTok}[1]{\textcolor[rgb]{0.13,0.29,0.53}{\textbf{#1}}}
\newcommand{\NormalTok}[1]{#1}
\newcommand{\OperatorTok}[1]{\textcolor[rgb]{0.81,0.36,0.00}{\textbf{#1}}}
\newcommand{\OtherTok}[1]{\textcolor[rgb]{0.56,0.35,0.01}{#1}}
\newcommand{\PreprocessorTok}[1]{\textcolor[rgb]{0.56,0.35,0.01}{\textit{#1}}}
\newcommand{\RegionMarkerTok}[1]{#1}
\newcommand{\SpecialCharTok}[1]{\textcolor[rgb]{0.00,0.00,0.00}{#1}}
\newcommand{\SpecialStringTok}[1]{\textcolor[rgb]{0.31,0.60,0.02}{#1}}
\newcommand{\StringTok}[1]{\textcolor[rgb]{0.31,0.60,0.02}{#1}}
\newcommand{\VariableTok}[1]{\textcolor[rgb]{0.00,0.00,0.00}{#1}}
\newcommand{\VerbatimStringTok}[1]{\textcolor[rgb]{0.31,0.60,0.02}{#1}}
\newcommand{\WarningTok}[1]{\textcolor[rgb]{0.56,0.35,0.01}{\textbf{\textit{#1}}}}
\usepackage{longtable,booktabs,array}
\usepackage{calc} % for calculating minipage widths
% Correct order of tables after \paragraph or \subparagraph
\usepackage{etoolbox}
\makeatletter
\patchcmd\longtable{\par}{\if@noskipsec\mbox{}\fi\par}{}{}
\makeatother
% Allow footnotes in longtable head/foot
\IfFileExists{footnotehyper.sty}{\usepackage{footnotehyper}}{\usepackage{footnote}}
\makesavenoteenv{longtable}
\usepackage{graphicx}
\makeatletter
\def\maxwidth{\ifdim\Gin@nat@width>\linewidth\linewidth\else\Gin@nat@width\fi}
\def\maxheight{\ifdim\Gin@nat@height>\textheight\textheight\else\Gin@nat@height\fi}
\makeatother
% Scale images if necessary, so that they will not overflow the page
% margins by default, and it is still possible to overwrite the defaults
% using explicit options in \includegraphics[width, height, ...]{}
\setkeys{Gin}{width=\maxwidth,height=\maxheight,keepaspectratio}
% Set default figure placement to htbp
\makeatletter
\def\fps@figure{htbp}
\makeatother
\setlength{\emergencystretch}{3em} % prevent overfull lines
\providecommand{\tightlist}{%
  \setlength{\itemsep}{0pt}\setlength{\parskip}{0pt}}
\setcounter{secnumdepth}{5}
\newlength{\cslhangindent}
\setlength{\cslhangindent}{1.5em}
\newlength{\csllabelwidth}
\setlength{\csllabelwidth}{3em}
\newlength{\cslentryspacingunit} % times entry-spacing
\setlength{\cslentryspacingunit}{\parskip}
\newenvironment{CSLReferences}[2] % #1 hanging-ident, #2 entry spacing
 {% don't indent paragraphs
  \setlength{\parindent}{0pt}
  % turn on hanging indent if param 1 is 1
  \ifodd #1
  \let\oldpar\par
  \def\par{\hangindent=\cslhangindent\oldpar}
  \fi
  % set entry spacing
  \setlength{\parskip}{#2\cslentryspacingunit}
 }%
 {}
\usepackage{calc}
\newcommand{\CSLBlock}[1]{#1\hfill\break}
\newcommand{\CSLLeftMargin}[1]{\parbox[t]{\csllabelwidth}{#1}}
\newcommand{\CSLRightInline}[1]{\parbox[t]{\linewidth - \csllabelwidth}{#1}\break}
\newcommand{\CSLIndent}[1]{\hspace{\cslhangindent}#1}
\definecolor{customcolor}{RGB}{32,178,170}
\colorlet{coverlinecolor}{customcolor}

% 下面如果不注释就准备好 Logo 和封面图片
% \logo{logo-blue.png} % 图片尺寸 1:1
% \cover{cover.jpg} % 图片尺寸 1280 × 1024

% Cancel common factors in Math 
\usepackage[makeroom]{cancel}

\usepackage[export]{adjustbox} %Needed for max width
%\patchcmd{<command>}{<code to replace>}{<code>}{<success>}{<failure>}
%The following codes add max dimension option to includegraphics
%Gin@ii is from graphicx package and looks for a second optional argument
\expandafter\patchcmd\csname Gin@ii\endcsname 
{\setkeys{Gin}{#1}}
{\setkeys{Gin}{max width=\textwidth, max height=.5\textwidth,keepaspectratio,#1}}
{}
{}

\definecolor{colortip}{RGB}{81,183,73}
\definecolor{colornote}{RGB}{251,188,5}
\definecolor{colorwarn}{RGB}{255,83,59}
\definecolor{colorinfo}{RGB}{204,204,204}

\tcbset{
  colbacktitle=white,
  enhanced,
  attach boxed title to top center={yshift=-2mm},
  colback=white, % 背景色
  coltext=black, % 文本色
  leftrule=1mm,
  rightrule=.25mm,
  bottomrule=.25mm,
  toprule=.25mm,
  boxsep=1pt, % 文字和边框的空隙
  arc=1pt % 圆角
}

\newtcolorbox{rmdtip}[1]{
  title=#1,
  coltitle=colortip,
  colframe=colortip, % 边框色
}

\newtcolorbox{rmdnote}[1]{
  title=#1,
  coltitle=colornote,
  colframe=colornote % 边框色
}

\newtcolorbox{rmdwarn}[1]{
  title=#1,
  coltitle=colorwarn,
  colframe=colorwarn % 边框色
}

\newtcolorbox{rmdinfo}{
  colframe=colorinfo % 边框色
}

\frontmatter
\usepackage[lotdepth=2, lofdepth=2]{subfig}
\usepackage[scale=0.85]{sourcecodepro}
\ifLuaTeX
  \usepackage{selnolig}  % disable illegal ligatures
\fi
\IfFileExists{bookmark.sty}{\usepackage{bookmark}}{\usepackage{hyperref}}
\IfFileExists{xurl.sty}{\usepackage{xurl}}{} % add URL line breaks if available
\urlstyle{same} % disable monospaced font for URLs
\hypersetup{
  pdftitle={纤维丛},
  pdfauthor={稻年; 叶 飞},
  colorlinks=true,
  linkcolor={Maroon},
  filecolor={Maroon},
  citecolor={Blue},
  urlcolor={Blue},
  pdfcreator={LaTeX via pandoc}}

\title{纤维丛}
\author{稻年 \and 叶 飞}
\date{}

\begin{document}
\maketitle

{
\hypersetup{linkcolor=}
\setcounter{tocdepth}{1}
\tableofcontents
}
\listoffigures
\listoftables
\mainmatter

\hypertarget{preface}{%
\chapter*{前言}\label{preface}}
\addcontentsline{toc}{chapter}{前言}

此笔记整理自(\protect\hyperlink{ref-CohenNotes}{Cohen 1998})

简单来说, 纤维丛 (fiber bundle) 就是在基空间 (base space) 的每点附着\textbf{同一个}\footnote{同一个指的是在同构意义下相同}纤维 (fiber) , 最经典的例子就是流形上的切丛: 在流形 \(M\) 上的每点附着一个线性空间, 也就是切空间 \(T_p M\), 所有这些切空间的不交并便是切丛 \(TM\). 对于一般的纤维丛而言, 基空间不一定是流形, 纤维也不一定是线性空间.

比纤维丛更一般的概念是\textbf{纤维化} (fibration) , 即所谓基空间的附着物不要求都一样, 这样就可以和代数几何中处理的奇异性问题相吻合.

纤维化是一个非常具有普适性的概念.

\begin{example}[集合的纤维化]
\protect\hypertarget{exm:preface-set-fiber}{}\label{exm:preface-set-fiber}例如对任意的集合映射 \[
f\colon A\to B,
\] 我们所说的纤维 \(f ^{-1}(b):=\{a\in A : f(a)=b\}\) 指的正是纤维化里的纤维: 我们对集合 \(B\) 中的每点 \(b\) 附着一个集合 \(f^{-1}(b)\); 而平常所说的满射在这就是指每处的纤维 \(f^{-1}(b)\) 都非空. 由此可见, 纤维化给了我们一个看待映射的全新视角, 即在陪域上 ``长'' 出一族纤维\footnote{当我们说 ``长出一丛纤维'' 时, 则默认该纤维化是纤维丛, 即每处长出同构的纤维}.
\end{example}

\begin{example}[复叠空间是纤维丛]
\protect\hypertarget{exm:preface-covering-fiber}{}\label{exm:preface-covering-fiber}复叠空间 (covering space) 其实就是拓扑空间上长出的一丛\footnote{注意这里的量词是 ``丛''}离散纤维, 即纤维是离散集合.
\end{example}

\begin{example}[纤维积是纤维化]
\protect\hypertarget{exm:preface-covering-fiber}{}\label{exm:preface-covering-fiber}纤维积 ( fiber product) 其实也是纤维化. 具体来说就是, 在范畴里给定两个态射 \(X\to Z \leftarrow Y\), 纤维积给出的正是基于此的 ``自然'' 对象. 用纤维化的观点来看, \(X,Y\) 都可以看成是 \(Z\) 的纤维化, 而纤维积所做的就是两件事: 既是某种意义上的乘积, 同时也保持给定的两个纤维化的结构.
\end{example}

\hypertarget{ux7ea4ux7ef4ux4e1bux7684ux5e94ux7528}{%
\section*{纤维丛的应用}\label{ux7ea4ux7ef4ux4e1bux7684ux5e94ux7528}}
\addcontentsline{toc}{section}{纤维丛的应用}

一般来说, 纤维丛和纤维化的目的是打包空间的拓扑信息与几何信息, 毕竟数学研究的就是如何获取并分析数学对象所含有的数学信息, 以此来理解纷繁复杂的数学现象. 下面举出这一思想在各个数学领域中的应用:

\begin{itemize}
\tightlist
\item
  \textbf{微分流形}: 许多结构都是在流形的切丛上构造的: 定向 (orientation), 活动标架 (framing), 殆复结构 (almost complex structure), 自旋结构 (spin structure) 以及黎曼度量 (Riemannian metric).
\item
  \textbf{代数拓扑}: 纤维化的同伦序列和谱序列一直是半个多世纪以来的基本工具.
\item
  \textbf{代数几何}: 一大基本问题便是理解仿射簇上代数丛 (algebraic bundle) 的代数截面 (algebraic section).
\item
  \textbf{\(K\)-理论}:
\item
  \textbf{微分拓扑}: 杨-米尔斯方程.
\end{itemize}

\hypertarget{locally-trivial-fibrations}{%
\chapter{局部平凡的纤维化}\label{locally-trivial-fibrations}}

考虑 Hausdorff 空间

\hypertarget{appendix}{%
\appendix}


\hypertarget{r-markdown}{%
\chapter{R Markdown}\label{r-markdown}}

This is an R Markdown document. Markdown is a simple formatting syntax for authoring HTML, PDF, and MS Word documents. For more details on using R Markdown see \url{http://rmarkdown.rstudio.com}.

When you click the \textbf{Knit} button a document will be generated that includes both content as well as the output of any embedded R code chunks within the document. You can embed an R code chunk like this:

\begin{Shaded}
\begin{Highlighting}[]
\FunctionTok{summary}\NormalTok{(cars)}
\end{Highlighting}
\end{Shaded}

\begin{verbatim}
##      speed           dist       
##  Min.   : 4.0   Min.   :  2.00  
##  1st Qu.:12.0   1st Qu.: 26.00  
##  Median :15.0   Median : 36.00  
##  Mean   :15.4   Mean   : 42.98  
##  3rd Qu.:19.0   3rd Qu.: 56.00  
##  Max.   :25.0   Max.   :120.00
\end{verbatim}

\hypertarget{including-plots}{%
\section{Including Plots}\label{including-plots}}

You can also embed plots, for example:

\begin{Shaded}
\begin{Highlighting}[]
\FunctionTok{par}\NormalTok{(}\AttributeTok{mar =} \FunctionTok{c}\NormalTok{(}\DecValTok{4}\NormalTok{, }\DecValTok{4}\NormalTok{, .}\DecValTok{1}\NormalTok{, .}\DecValTok{1}\NormalTok{))}
\FunctionTok{plot}\NormalTok{(pressure)}
\end{Highlighting}
\end{Shaded}

\begin{figure}

{\centering \includegraphics[width=0.8\linewidth]{07-appendix_files/figure-latex/nice-fig-02-1} 

}

\caption{Here is another nice figure!}\label{fig:nice-fig-02}
\end{figure}

Note that the \texttt{echo\ =\ FALSE} parameter was added to the code chunk to prevent printing of the R code that generated the plot.

\hypertarget{References}{%
\chapter*{参考文献}\label{References}}
\addcontentsline{toc}{chapter}{参考文献}

\hypertarget{refs}{}
\begin{CSLReferences}{1}{0}
\leavevmode\vadjust pre{\hypertarget{ref-CohenNotes}{}}%
Cohen, Ralph L. 1998. \emph{The Topology of Fiber Bundles Lecture Notes}. Stanford University. \url{http://math.stanford.edu/~ralph/fiber.pdf}.

\end{CSLReferences}

\end{document}
